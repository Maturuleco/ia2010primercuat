\section{Descripcion}
%Descripción y desarrollo de los experimentos.

El dasarrollo del sistema se puede dividir en cuatro etapas: \emph{armado del diccionario}, \emph{definicion de reglas}, \emph{aplicacion de reglas} y \emph{recoleccion de resultados}. Las mismas serás descriptas en más profundidas a continuación.

\subsection{Armado del diccionario}

Antes de empezara a analizar los comentarios, fué nescesario definir las palabras que definen nuestro mundo y caracterizarlas, ya que por ejemplo, no cumplen la misma función gramátical \emph{pizza} o \emph{exelente}.

Las entidades gramáticales que se terminaron usando son:
\begin{itemize}
\item Entidad: Define a los sustantivos del dominio relevantes, pueden ser comidas o mesas, ambiente, etc.
\item Adjetivo: Son los adjetivos calificativos, estos a su vez se dividieron en:
\subitem Positivo
\subitem Neutro
\subitem Negativo
\item Verb: verbos
\item Art: articulos
\item Empty: Son todas aquellas palabras que no están definidas como ninguna otra entidad gramátical.
\end{itemize}

En este punto, hubiera sido ideal poder definir todas las palabras del idioma castellano, pero esto no es posible, por lo que se hizo fué definir las palabras más relevantes en nuestro dominio (los comentarios de la guia oleo).

Con este fin fué nesesario primero determinar cuáles son estas palabras, lo que se optó por hacer fué simplemente ordenar las palabras por cantidad de apariciones en los comentarios. Con este procedimiento se logró conseguir una primera aproximación de las palabras más relevantes.

Pero dado que el tiempo con que se contaba no era suficiente para caracterizar todas las palabras, se optó por caracterizar las palabras por su raíz, esto disminuyó notablemente la cantidad de palabras, ya que para caracterizar palabras como \emph{malo, mala, mal, malos, malas, males} sólo fué nescesario caracterizar \emph{mal}.

Este método si bien ayuda a disminuir la cantidad de elementos del dominio, también trae sus consecuencias, ya que palabras como \emph{precios} y \emph{preciosamente} tienen la misma raíz \emph{preci}, siendo la primera una \emph{Entidad} y la otra un \emph{Adjetivo}. Pero esto fué algo con lo que se aceptó seguir adelante.

Una vez obtenido un diccionario con todas las raices ordenadas por orden de aparición se comenzaron a caracterizar las más importantes manualmente. Es necesario aclarar que la caracterización de las palabras fué considerando el contexto de la \emph{GuíaOleo}, ya que palabras como \emph{tierno} se tomó como un adjetivo positivo y palabras como \emph{abuelo} no se consideraron como entidades a pesar de ser sustantivos.

Luego de completar las subsiguientes etapas, una vez que se obteubo un sistema andando con este diccionario, se decidió enriquecer el diccionario de un modo más automático. Este proceso será explicado más adelante.


\subsection{Definición de reglas}

Una vez obtenido el diccionario de palabras a usar, el siguiente paso fué representar el criterio por el cuál se definiría si una frase es relevante o no. Esto se realizó mediante el uso de expresiones regulares.

De esta manera se definieron expresiones regulares de modo tal que sólo las frases que cumplan con estas expresiones cumplan con el criterio propuesto. Esto fué hecho manualmente y las expresiones regulares se fueron creando a base de sentido común y pruebas en el conjunto de datos de entrenamiento.

Las reglas que se obtubieron finalmente fueron las siguientes:

\begin{verbatim}
(ART)* ENTITY (EMPTY)+ (ADJETIVE)+ ( y |ADJETIVE)*
(ART)* ENTITY (VERB)+ (ADJETIVE)+ (EMPTY|ADJETIVE)*
(ART)* ENTITY (VERB)+ (EMPTY)+ (ADJETIVE)+ (EMPTY|ADJETIVE|ART)* (ENTITY)*
(ART)+ (ADJETIVE)+ ENTITY (EMPTY)+ (ADJETIVE)+
(ART)+ ENTITY (ADJETIVE)+
(no)* (VERB)* (ART)+ (ADJETIVE)+ ENTITY
(ART)* ADJETIVE ART ENTITY
(ART)* VERB ADJETIVE ART ENTITY
VERB VERB ADJETIVE
Muy ADJETIVE (ART)* ENTITY
uno de los ADJETIVE ENTITY para comer
no lo recomiendo
no es para recomendar
\end{verbatim}

Se puede ver que la mayoría de las reglas depende sólo de las entidades definidas, pero hay casos en los que la expresion de la caracterización no es suficiente y se pierden casos de gran interés, como ser \comentario{no lo recomiendo}. Para no perder esto simplemente se agregó una regla que busca esta frase en particular.

En este punto es justo hacer notar que los modificadores tales como \emph{muy}, \emph{no}, \emph{más}, etc. no fueron caracterizados, haciendo imposible la detección de estas palabras en frases más complejas  Al igual que no se detectan frases que contengan aposiciones u otras construcciones gramáticas con relativo grado de complejidad. Estos casos quedan para futuras iteraciones de este sistema.

\subsection{Aplicación de reglas}

Una vez que se cuenta con el diccionario de las palabras ya caracterizado y con las reglas definidas, es necesario aplicar estas reglas a los comentarios. Con este fín se desarrolló un sistema informático que simplemente busca la aparicion de las expresiones regulares en un comentario.

El mismo toma como entrada:
\begin{itemize}
\item los diferentes archivos con cada tipo de palabras,
\item un archivo con las reglas definidas y
\item los comentarios a evaluar.
\end{itemize}

Este devuelve un archivo de texto con todos los comentarios seguidos de las frases obtenidas y un detalle de cuál fué la regla que se usó para detectar la frase.

\subsection{Recolección de resultados}
% Positivos, negativos, bla...

Los resultados obtenidos se plasmaron en un archivo de texto plano, en los mismos se encuentran el comentario original analizado y las reglas que aplicaron en el mismo, seguidas de el extracto que aplicó.
De esta manera se puede observar con claridad el funcionamiento del sistema de extracción de información.

Por otro lado, en base a los resultados obtenidos se realizó un análisis preliminar sobre cada uno, esto fué posible gracias a que los adjetivos fueron divididos en tres clases
\begin{itemize}
\item Positivos,
\item Negativos,
\item Neutros
\end{itemize}
De esta manera ante un comentario se hizo un balance entre los adjetivos positivos y negativos que el mismo contiene para dar una idea preliminar del matiz del mismo.

Este resultado puede ser ampliamente mejorado, ya que no se vieron aspectos como la relación entre los diferentes adjetivos o el contexto en el que se encuentran los mismos ni el peso de estos. Ya que es muy distinto el peso de un adjetivo como \emph{exelente} y un adjetivo como \emph{bueno}, siendo los dos adjetivos positivos.

Una vez realizado esta categorización se realiza también un resumen del restaurant teniendo en cuenta todos los comentarios del mismo y el aspecto general de cada uno de ellos.

Se decidió este acercamiento, en vez de tener en cuenta el porcentaje de adjetivos positivos en el total de los comentarios, porque lo que se busca es encontrar el porcentaje de gente que estubo conforme con el restaurant, por esto es que primero se analiza cada comentario por separado y se lo clasifica aislado del resto y luego se clasifica el restaurant en base a los mismos.

Vale aclarar que este análisis de los datos es sólo una análisis superficial, ya que el mismo excede el alcance del trabajo propuesto.

\subsection{Consideraciones finales}
% Be free ;)
