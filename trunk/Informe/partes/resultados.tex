\section{Resultados}
%Presentación y análisis de resultados.

Los resultados obtenidos lograron identificar en gran medida las partes relevantes de los comentarios, a continuación se analizaran distintos ejemplos:

\begin{quotation}

\emph{
\lq{}\lq{}Hacia tiempo que no iba y me agarro nostalgia.  Los sabores siguen identicos, {\bf la cocina es excelente} y la pizza, a mi entender, de {\bf la mejor pizza} italiana que se hace en Buenos Aires, finita, bien a la piedra, y con ingredientes de primera.  El carpaccio de {\bf lomo es buenisimo} y da para compartir.  Creo que es la mejor opcion cuando uno esta por el centro.  Comida italiana genuina.  Muy recomendable. \rq{}\rq{}
}

Resultado:
\begin{itemize}
\item R1  la cocina es excelente- lomo es buenisimo
\item R5  la mejor pizza
\end{itemize}

\end{quotation}

Lo primero que se puede apreciar es que las partes identificadas como importantes realmente lo son. Veamos que \comentario{la cocina es excelente}, \comentario{lomo es buenisimo} y \comentario{la mejor pizza}, son tres apreciaciones sobre el restaurant en sí. En cambio oraciones como \comentario{Hacia tiempo que no iba y me agarro nostalgia}, no lo es, sino que es una comentario personal. Este último tipo de oraciones son las que se desan obviar.

\begin{quotation}

\emph{
\lq{}\lq{}Fuimos varios a almorzar y {\bf nos atendieron correctamente}, {\bf la comida es muy rica} y {\bf el ambiente es descontracturado} y casual.  En resumidas cuentas, {\bf un buen lugar} para {\bf comer una rica} pizza, si estan por la zona de Retiro.\rq{}\rq{}
}


Resultado:
\begin{itemize}
\item R0  comer una rica-
\item R1  el ambiente es descontracturado-
\item R2  la comida es muy rica-
\item R5  un buen lugar-
\item R8  nos atendieron correctamente-
\end{itemize}

\end{quotation}

Veamos que este comentario se reconoció casi en su totalidad como relevante. Se puede ver que el dato de que el restaurant queda en la zona de retiro no se identifico, esto se debe a que la localización del restaurant no cumple con el criterio propuesto para el sistema.

Otra cosa relevante a ver es que se reconocío \comentario{un buen lugar} y \comentario{comer una rica} como frases relevantes por separado, esto se debe a que los conectores como \emph{para} no se incluyeron en el diccionario de palabras.

Por otro lado se reconoció como una oración relevante: \comentario{comer una rica}, la regla que se aplicó para identificarla es:

\begin{verbatim}
R0: (ART)* ENTITY (EMPTY)+ (ADJETIVE)+ (y|ADJETIVE)*;
\end{verbatim}

Esto se debe a que la palabra \emph{comer} se reconoció como una entidad y no como un verbo, ya que su raíz es \emph{com}, que también lo es para la palabra \comentario{comida}, y al ser está ultima una entidad tan importante en el contexto del trabajo se decidio identificarla como tal.
