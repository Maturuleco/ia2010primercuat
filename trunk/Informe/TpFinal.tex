%%	SECCION documentclass																									 %%	
%%---------------------------------------------------------------------------%%
\documentclass[a4paper]{report}

%\usepackage{amsmath}
%\usepackage{amsfonts}
%\usepackage{amssymb}

%%---------------------------------------------------------------------------%%
%%	SECCION usepackage																											 %%	
%%---------------------------------------------------------------------------%%
\usepackage{amsmath, amsthm}
\usepackage{amsfonts}%
\usepackage{amssymb}%
\usepackage{caratula}
\usepackage{hyperref} %para que haya hiperv�nculos
\usepackage{graphicx} % Para el logo magico!
\usepackage{alltt} % que es este paquete???
\usepackage{tabularx}

\usepackage[spanish]{babel} 
\usepackage[T1]{fontenc}
\usepackage{textcomp}
\usepackage[utf8]{inputenc}

\pagenumbering{roman}
\hyphenation{biggest-satelit bi-gestsatelit biggestsa-telit biggestsate-lit}
\newenvironment{seccion}[1]{\section*{\underline{#1}}\linea \begin{flushleft}}{\end{flushleft}}

\oddsidemargin 0cm
\headsep -1.4cm
\textwidth=16.5cm
\textheight=23cm
%\makeindex



%%---------------------------------------------------------------------------%%
%%	SECCION document	                                                       %%	
%%---------------------------------------------------------------------------%%
\begin{document}
	\setcounter{page}{0} %para que numere a la caratula como p�gina 0 y a las demas a partir de 1

%%---- Caratula -------------------------------------------------------------%%
\title{Trabajo Práctico IA2}
\author{Matías Pérez\\ Cecilia Sanchez}
\date{Agosto, 2010}

\materia{Inteligencia Artificial}
\titulo{Extracción de la Informacion}
\subtitulo{SEI-GO}
\fecha{31 Mayo, 2010}

\integrante{Cecilia Sanchez}{000/00}{chechus26@gmail.com}
\integrante{Matías Pérez}{002/05}{elmaildematiaz@gmail.com}
%\resumen{}

\maketitle

\tableofcontents

\newpage

\clearpage
%Introducción. Objetivo y alcance.
\section{Introduccion}

\subsection{Obejtivo}

El trabajo consiste en la realización de un sistema de extracción de información (SEI) para la \emph{GuiaOleo}\footnote{http://www.guiaoleo.com.ar}, el cuál consiste en obtener información relevante a partir de los comentarios que se encuentran en la misma.

El criterio usado para definir si cierta información es relevante o no, es si la misma consiste en apreciaciones sobre la comida, el servicio y otros aspectos de los restaurants. De esta forma, se considera información irrelevante aquella que trate sobre la persona en sí o información que no esté relacionada con el lugar. Como ser \comentario{Voy siempre en navidad} o \comentario{Fuí con mi pareja}.

%Vale aclarar que dicho criterio es arbitrario, ya que se hubiera podido elegir cualquier otro criterio.
Dado el domino del sistema, se optó por este criterio, pero en otro contexto, el mismo puede no tener sentido.

El objetivo práctico del SEI es conseguir de manera sencilla y rápida las carácterísticas relevantes de los restaurants sin leer la totalidad de los comentarios, esto es de gran utilidad si se cuenta con un gran volumen. Es imporatante destacar que si el sistema funciona correctamente la cantidad de información a analizar es mucho menor. 
%Se trata de obtener de los comentarios los fragmentos de los mismos que cumplen con el criterio ante dicho.

Para el siguiente comentario:
\begin{quotation}
\emph{\lq\lq{}Es uno de los pocos restaurantes que sirve cocina italiana de verdad y actualiza su menu; la atencion, si bien llevada mayormente por chicas jovenes es sobria, amable y eficiente. Voy muy seguido y he celebrado Navidad alli. Recomiendo tanto las pizzas como los platos de cocina. Dejen lugar para los postres...valen la pena acompañados por un ristretto lavazza autentico.\rq\rq{}}
\end{quotation}

Los resultados esperados sería obtener las oraciones:

\comentario{Recomiendo tanto las pizzas como los platos de cocina} y \comentario{la atencion, [..] es sobria, amable y eficiente}.

y obviar las oraciones como \comentario{Voy muy seguido y he celebrado Navidad alli} ya que no expresa información relevante.


\subsection{Alcance}

En esta primera versión del sistema, el trabajo se enfocó en encontrar solamente oraciones que cumplan con ciertos patrones del lenguaje natural. Dado que los comentarios extraidos de la página corresponden a texto libre, muchas oraciones que contienen información importante se pueden perder ya que la manera en que se encuentran expresadas puede ser muy diversa.

Por ejemplo, no recomendar un lugar se puede expresar de distintas maneras:
\begin{itemize}
\item \comentario{Les recomiendo no ir}
\item \comentario{No le recomiendo ir a nadie}
\item \comentario{Yo no lo recomendaria}
\item \comentario{El lugar no es nada recomendable}
\end{itemize}

Teniendo un conjunto acotado de reglas quizás no todos los casos mensionados anteriormente puedan ser reconocidos. Estos instancias se verán incluidas en las siguientes refinaciones del sistema.





%Descripción y desarrollo de los experimentos.
\section{Descripcion}
%Descripción y desarrollo de los experimentos.

El desarrollo del sistema se puede dividir en cuatro etapas: \emph{armado del diccionario}, \emph{definicion de reglas}, \emph{aplicacion de reglas} y \emph{recoleccion de resultados}. Las mismas serás descriptas en más profundidas a continuación.

\subsection{Armado del diccionario}

Antes de empezar a analizar los comentarios, fué nescesario definir las palabras relevantes del dominio y clasificarlas, ya que por ejemplo, no cumplen la misma función gramátical \emph{pizza} o \emph{excelente}.

Las palabras se clasificaron en los siguientes grupos: 
\begin{itemize}
\item Entidad: Define a los sustantivos relevantes del dominio, como ser comidas, atención, ambiente, etc.
\item Adjetivo: Son los adjetivos calificativos, estos a su vez se dividieron en:
\subitem Positivo
\subitem Neutro
\subitem Negativo
\item Verbo: verbos
\item Artículo: articulos
\item Palabras vacías: todas aquellas palabras que no están clasificadas en ningun otro grupo.
\end{itemize}

Dado que el idioma castellano contiene un gran volumen de palabras, no es posible clasificarlas en su totalidad. Con lo que se definió un diccionario acotado de palabras relevantes en nuestro dominio (los comentarios de la guia oleo).

El primer paso para construir el diccionario fue simplemente ordenar por cantidad de apariciones las palabras de nuestro conjunto de datos. Con este procedimiento se logró conseguir una primera aproximación sobre la importancia de las palabras. Con esta lista ordenada se tomaron las primeras 1000 como primer diccionario. Esto se debe a que la clasificación en grupos es de manera manual y lleva mucho tiempo.

Pero dado que el tiempo con que se contaba no era suficiente para caracterizar todas las palabras, se realizó una nueva iteración sobre estas para obtener solamente la raíz para así disminuir el volumen a clasificar y poder obtener un diccionario mas completo. Por ejemplo, para caracterizar palabras como \emph{malo, mala, mal, malos, malas, males} sólo fué nescesario caracterizar \emph{mal}.

Este método si bien ayuda a disminuir la cantidad de elementos del dominio, también trae sus consecuencias, ya que palabras como \emph{precios} y \emph{preciosamente} tienen la misma raíz \emph{preci}, siendo la primera una \emph{Entidad} y la segunda un \emph{Adjetivo}. Pero esto fué algo con lo que se aceptó seguir adelante.

Una vez obtenido un diccionario con todas las raices ordenadas por cantidad de apariciones se comenzaron a caracterizar solamente las más importantes. Es necesario aclarar que la clasificación se realizó en el contexto de la \emph{GuíaOleo}, ya que palabras como \emph{tierno} se tomó como un adjetivo positivo y palabras como \emph{abuelo} no se consideraron como entidades a pesar de ser sustantivos.

Una vez que se desarrollo el sistema con este diccionario, se decidió enriquecerlo de manera semiautomática. Este proceso será explicado más adelante.


\subsection{Definición de reglas}

Con un diccionario definido, el siguiente paso fué representar los criterios por los cuales que clasificaría a una oración como relevante o no. Esto se realizó mediante el uso de expresiones regulares.

De esta menara se definieron expresiones regulares para la busqueda de las oraciones que cumplen con el criterio mensionado anteriomente. Esto fué realizado manualmente y las expresiones regulares se fueron creando en base al sentido común y pruebas sobre conjunto de datos de entrenamiento.

Las reglas que se obtubieron finalmente fueron las siguientes:

\begin{verbatim}
(ART)* ENTITY (EMPTY)+ (ADJETIVE)+ ( y |ADJETIVE)*
(ART)* ENTITY (VERB)+ (ADJETIVE)+ (EMPTY|ADJETIVE)*
(ART)* ENTITY (VERB)+ (EMPTY)+ (ADJETIVE)+ (EMPTY|ADJETIVE|ART)* (ENTITY)*
(ART)+ (ADJETIVE)+ ENTITY (EMPTY)+ (ADJETIVE)+
(ART)+ ENTITY (ADJETIVE)+
(no)* (VERB)* (ART)+ (ADJETIVE)+ ENTITY
(ART)* ADJETIVE ART ENTITY
(ART)* VERB ADJETIVE ART ENTITY
VERB VERB ADJETIVE
Muy ADJETIVE (ART)* ENTITY
uno de los ADJETIVE ENTITY para comer
no lo recomiendo
no es para recomendar
\end{verbatim}

Se puede ver que la mayoría de las reglas depende sólo de las entidades definidas, pero hay casos en los que la expresion de la caracterización no es suficiente y se pierden casos de gran interés, como ser \comentario{no lo recomiendo}. Para no perder esto simplemente se agregó una regla que busca esta frase en particular.

En este punto es justo hacer notar que los modificadores tales como \emph{muy}, \emph{no}, \emph{más}, etc. no fueron caracterizados, haciendo imposible la detección de estas palabras en frases más complejas  Al igual que no se detectan frases que contengan aposiciones u otras construcciones gramáticas con relativo grado de complejidad. Estos casos quedan para futuras iteraciones de este sistema.

\subsection{Aplicación de reglas}

Una vez que se cuenta con el diccionario de las palabras clasificadas y con las reglas, es necesario aplicarlas a los comentarios. Con este fín se desarrolló un sistema informático que realice esta tarea.

El sistema toma como entrada:
\begin{itemize}
\item un archivo por cada grupo de palabras,
\item un archivo con las reglas definidas y
\item los comentarios a evaluar.
\end{itemize}

El resultado del procesamiento es un archivo de texto con todos los comentarios seguidos de las frases obtenidas y un detalle de cuál fué la regla que se usó para detectar la frase.
Por ejemplo:

\begin{quotation}
\emph{
\lq{}\lq{}{\bf La ambientacion es piola aunque} es {\bf un lugar bastante ruidoso}. {\bf  La pizza sigue siendo rica}. La desilusion fue el tartufo, que la moza me aseguro que era igual que en Italia y era un simple helado de chocolate.  Que lo llamen de otra manera, entonces.\rq{}\rq{}
}

Resultado:
\begin{itemize}
\item R0  un lugar bastante ruidoso - la pizza sigue siendo rica
\item R1  la ambientacion es piola aunque
\end{itemize}

\end{quotation}

\subsection{Recolección de resultados}
% Positivos, negativos, bla...

El siguiente es un ejemplo del resultado completo de sistema aplicado a un comentario:
\begin{verbatim}
comment: POSITIVO
<1-ops.yaml, user: 17248, service: bueno, food: buena, enviroment: muy bueno, date: 21-05-2010>
Texto orig: La ambientacion es piola aunque es un lugar bastante ruidoso.  La pizza sigue siendo rica. La desilusion fue el tartufo, que la moza me aseguro que era igual que en Italia y era un simple helado de chocolate.  Que lo llamen de otra manera, entonces. 
	result:
               | R0  un lugar bastante ruidoso- la pizza sigue siendo rica-
               | R1  la ambientacion es piola aunque-
\end{verbatim}
               
Donde lo primero que se lista es la clasificación del comentario como positivo/negativo (esta clasificación será explicada con mas detalle mas adelante), las clasificaciones de servicio, comida, ambiente realizadas por el usuario, el comentario original y el resultado de la aplicación de las reglas.

De esta manera se puede observar con claridad el funcionamiento del sistema de extracción de información.

En base a los resultados obtenidos se realizó un análisis preliminar sobre cada comentario, esto fué posible gracias a que los adjetivos fueron divididos en tres tipos
\begin{itemize}
\item Positivos,
\item Negativos,
\item Neutros
\end{itemize}

De esta manera dado un comentario se hizo un balance entre los adjetivos positivos y negativos que se identificaron como relevantes para dar una idea preliminar del matiz del mismo.

Una vez realizada la categorización de todos los comentarios de un restaurant se creo también un resumen de éste, para tener un una balance general de opiniones.

Para clasificar al restaurant se decidió este acercamiento, en vez de tener en cuenta el porcentaje de adjetivos positivos/negativos en el total de los comentarios, porque lo que se buscó fue encontrar el porcentaje de gente que estuvo conforme con el restaurant, contando a cada persona como un voto positivo o negativo según el matiz general de su comentario.

Vale aclarar que este análisis de los datos es sólo una análisis superficial, ya que un análisis mas profundo excede el alcance del trabajo propuesto.

\subsection{Consideraciones finales}
% Be free ;)

Para empezar, veamos un ejemplo de resultados:

\begin{quotation}

\emph{
\lq{}\lq{}{\bf El ambiente es agradable} y sencillo, {\bf la atencion fue muy buena}, de hecho ayudaron con sugerencias y recomendaciones ( quizas por ser dia de semana ) .  {\bf La comida fue deliciosa} y mi acompaniante, que estuvo en Mexico 2 meses, aseguro que {\bf fue el mejor Chile} Relleno que probo en Bs As. La pasta de frijoles y el guacamole exquisitos. {\bf La copa grande} de {\bf margarita es muy rica}. Creo que es el mas fiel a Mexico de todos los restaurantes, ya que tiene el estilo y sencillez de comida y el ambiente que realmente existe en ese pais.\rq{}\rq{}
}

Resultado:
\begin{itemize}
\item R1  el ambiente es agradable - la comida fue deliciosa
\item R2  la atencion fue muy buena - margarita es muy rica
\item R4  la copa grande
\item R5  fue el mejor chile
\end{itemize}

\end{quotation}

En este ejemplo se puede observar que lo único que falta identificar es la frase \comentario{La pasta de frijoles y el guacamole exquisitos}. Esto se debe a que tanto \emph{guacamole} como \emph{frijoles} no están identificados en el diccionario.

Esta falla es un caso típico que se dá cuando se introduce a la base de comentarios uno que haga referencia a platos no identificados previamente. Por este motivo es que se propuso una nueva forma de identificar entidades del dominio de manera más dinámica.

El proceso realizado para esto fue identificar a partir de resultados obtenidos frases relevantes. Luego se llevaron a reglas las estructuras de estas frases, y se procesan los comentarios, esto se verá mejor con un ejemplo:

Se toma la frase \comentario{fue el mejor chile}, que tiene la siguiente estructura \begin{verbatim}VERBO ART ADJETIVO ENTIDAD\end{verbatim}, si lo que se quiere es reconocer todas las entidades que no se encuentran en el diccionario se remplaza ENTIDAD por la palabra clave \begin{verbatim}XXXX\end{verbatim}, de esta manera la regla sería: \begin{verbatim}VERBO ART ADJETIVO XXXX\end{verbatim}. 

Al analizar los comentarios del nuevo restaurant con esta regla, el sistema busca todas las oraciones que coincidan de manera tal que \begin{verbatim}{XXXX\end{verbatim} sea cualquier palabra no conocida, reportandola como posible ENTIDAD.

Existen casos en las que no se desean identificar palabras especificas, por esta razón se incorporó un diccionario de excepciones, las cuales no serán reconocidas con el método anterior.

%Metodologías y métricas utilizadas.
\section{Metodología}

\subsection{Sub-seccion1}
Bla,blalb,a
\subsection{Sub-seccion2}
BlaBla
\begin{flushleft}
Escribo sin tab
\end{flushleft}
Quiero enfatizar esta \emph{palabra}
\begin{itemize}
\item Item 1
\item Item 2
\subitem sub-item1
\subitem sub-item2
\item otro item
\end{itemize}










%Presentación y análisis de resultados.
\section{Resultados}
%Presentación y análisis de resultados.

Los resultados obtenidos lograron identificar en gran medida las partes relevantes de los comentarios, a continuación se analizaran distintos ejemplos:

\begin{quotation}

\emph{
\lq{}\lq{}Fue una experiencia espantosa, la pizza parecia una galletita de agua con tomate rebajado en agua unas hojitas de radicheta y unos pedacitos infimos de queso brie.  de sabor horrenda,precio carisimo, tres pizzas grandes y en promedio dos coronas chicas por persona casi 500 pesos  ( eramos 6 personas ) .  les recomiendo abstenerce de ir, {\bf no es un buen lugar}, no se come bien, {\bf el servicio es malisimo}, te ponen la botella de corona en la mesa sin siquiera servirte medio vaso como corresponde.  un asco, deberia darles verguenza tener un restaurant asi, el peor al que he ido.\rq{}\rq{}
}


Resultado:
\begin{itemize}
\item R1  el servicio es malisimo
\item R5  no es un buen lugar
\end{itemize}

\end{quotation}

Lo primero que se puede apreciar es que las partes identificadas como importantes realmente lo son. Veamos que \comentario{el servicio es malisimo} y \comentario{no es un buen lugar}, ambos son dos apreciaciones sobre el restaurant en sí. En cambio partes como \comentario{les recomiendo abstenerse de ir}, no lo es, sino que es una comentario personal. Este último tipo de frases son las que se desan obviar.

Por otro lado también se puede ver que se escapan frases que resultarían relevantes, como ser \comentario{no se come bien} o \comentario{precio carísimo}, esto se debe a la falta de presicion en las reglas. Se intentaron incluir este tipo de frases e incluir más reglas, pero muchas veces se generaba que se identifiquen como relevantes frases que no lo son, así que se optó por no identificar frases de más ante el peligro de obviar frases relevantes.


\begin{quotation}

\emph{
\lq{}\lq{}Fuimos varios a almorzar y {\bf nos atendieron correctamente}, {\bf la comida es muy rica} y {\bf el ambiente es descontracturado} y casual.  En resumidas cuentas, {\bf un buen lugar} para {\bf comer una rica} pizza, si estan por la zona de Retiro.\rq{}\rq{}
}


Resultado:
\begin{itemize}
\item R0  comer una rica-
\item R1  el ambiente es descontracturado-
\item R2  la comida es muy rica-
\item R5  un buen lugar-
\item R8  nos atendieron correctamente-
\end{itemize}

\end{quotation}

Veamos que en este comentario se reconoció casi en su totalidad como relevante. Lo único que se dejó afuera del mismo es que queda en la zona de retiro, esto se debe a que la localización del restaurant no cumple con el criterio propuesto para el sistema.

Otra cosa relevante a ver es que se reconocío \comentario{un buen lugar} y \comentario{comer una rica} como frases relevantes por separado, esto se debe a que los conectores como \emph{para} no se incluyeron en el diccionario de palabras.

Por otro lado se reconoció como una buena frase a \comentario{comer una rica}, la regla que se aplicó para identificarla es:

\begin{verbatim}
R0: (ART)* ENTITY (EMPTY)+ (ADJETIVE)+ (y|ADJETIVE)*;
\end{verbatim}

Lo que sucedió aquí es que \emph{comer} se reconoció como una entidad, esto se debe a que la raíz de comer es \emph{com}, y esta es raíz también de: \comentario{comida}, y al ser está ultima una entidad tan importante en el contexto del trabajo se optó por identificarla como entidad bajo el riesgo de confundirla con el verbo \emph{comer}.
%Conclusiones.
\section{Conclusiones}

Los resultados obtenidos por el sistema implmentados fueron más que satisfactorios. Se puedo apreciar cómo con simples reglas se puede identificar, a apartir de texto libre en lenguaje natural, contenidos relevantes de contenidos fútiles bajo cierto criterio.

Cuando se comnzó con el trabajo de investigación no se tenían grandes expectativas para con el mismo. Pero a medida que el mismo fué creciendo y perfeccionandose se pudo ver cómo se puede crear un sistema funcional que cumpla con el propósito planteado. A medida que se fué iterando tanto sobre las reglas que definen el criterio de relevancia de una frase, como sobre el armado del diccionario de las palabras del dominio se pudo ver como estas refinaciones perfeccionaron el mismo. El progreso del sistema se vió plasmado claramente en los resultados que se fueron viendo.

En las primeras iteraciones a partir de un comentario como:

\begin{quotation}
\emph{
\lq\lq{}Para redondear, todo es de mal gusto, salvo la camarera que nos atendio bien, el resto desastre, precios altisimos, la comida no es mas de lo que se puede comer en un bodegon italiano, recomiendo que no vayan, van a salir de mal humor.\rq\rq{}
}

\end{quotation}

Se obtenían resultados como:

\begin{quotation}
\begin{itemize}
\item R0  bodegon italiano , recomiendo
\item R3  el resto desastre
\end{itemize}
\end{quotation}

Luego, al ir iterando sobre el sistema y a medida que se detectaron las falencias del mismo fué posible que a partir del mismo comentario el sistema detectara correctamente las oraciones relevantes. En el caso planteado anteriormente luego de refinar las reglas y el diccionario fué posible identificar como irrelevante \comentario{bodegon italiano , recomiendo}, con lo que el resultado obtenido finalmente fué sólo el siguiente:

\begin{quotation}
\begin{itemize}
\item R4  el resto desastre
\end{itemize}
\end{quotation}

Veamos otro ejemplo; en una de las primeras iteraciones, ante el siguiete comentario:

\begin{quotation}
\emph{
\lq\lq{} {\bf La pizza nos encanta}, aunque esta vez vino medio fria y bastante quemada. El servicio de mesa es caro (para dos bruschettas de tomate). {\bf La bebida, carisima} y los postres, tambien {\bf son muy caros} dada la calidad. En fin, {\bf la pizza es excelente}, lo demas, deja bastante que desear. {\bf EL servicio es pesimo}, y bastante lento: le pedimos a la moza que nos limpiara, por favor, {\bf la mesa y lo hizo de muy mala gana}. El mantenimiento de los baños es poco.\rq\rq{}
}

\end{quotation}

Se obtenían resultados como:

\begin{quotation}
\begin{itemize}
\item R0  La bebida , carisima 
\item R0 son muy caros
\item R0 la mesa y lo hizo de muy mala
\item R1 La pizza nos encanta
\item R1 mesa es caro
\item R1 la pizza es excelente
\item R1 El servicio es pesimo
\end{itemize}
\end{quotation}

Pero luego se consiguió oibtener el siguiente resultado:

\begin{quotation}
\emph{
\lq\lq{} {\bf La pizza nos encanta}, aunque esta vez vino medio fria y bastante quemada. El servicio de {\bf mesa es caro} (para dos bruschettas de tomate). La bebida, carisima y los postres, tambien son muy caros dada la calidad. En fin, {\bf la pizza es excelente}, lo demas, deja bastante que desear. {\bf EL servicio es pesimo}, y bastante lento: le pedimos a la moza que nos limpiara, por favor, la mesa y lo hizo de muy mala gana. El mantenimiento de los baños es poco.\rq\rq{}
}

\end{quotation}

Obteniendo resultados más precisos, que es el objetivo que se planteó. Ya que como se explicó en un comienzo se prioriza no identificar información irrelevante como relevante, a riesgo de obviar información que resulte relevante.

Aunque se puede notar que el sistema no es perfecto, sí se puede decir que considerando la totalidad de los casos, el mismo identifica correctamente las partes relevantes en las mayorías de estos.

Fuera de las iteraciones que se realizaron hasta el momento del sistema quedan casos que tiene que ver con la gramática, el contexto o frases verbales. De igual manera resulta interesante ver cómo sólo teniendo en cuenta las oraciones en sí y su configuración se pueden lograr resultados ampliamente satisfactorios, como ser:

\begin{quotation}
\emph{
\lq\lq{}
Filo continua sirviendo platos de {\bf la cocina italiana de muy buena calidad}: {\bf la pizza estaba deliciosa}; el rissoto hecho en el momento, con funghi de primera y arroz arborio, estaba  impecable; el semifredo final fue un poema.  {\bf La atencion muy buena}, especialmente en un lugar con tanto movimiento como es Filo, {\bf La decoracion descontracturada} y transgresora muy acorde con el sitio.  En resumen un restaurante de {\bf comida italiana de muy buen nivel} en el centro de BA.  Para mi gusto, {\bf un lugar muy recomendable}, que me da la seguridad de encontrar en el lo que busco cuando quiero comida italiana. 
\rq\rq{}
}
Resultado:
\begin{itemize}
\item R0 la cocina italiana de muy buena calidad
\item R0 la pizza estaba deliciosa
\item R0 la atencion muy buena
\item R0 comida italiana de muy buen nivel
\item R0 un lugar muy recomendable
\item R4  la decoracion descontracturada
\end{itemize}
\end{quotation}



%Trabajo Futuro y Referencias.
\section{Trabajo Futuro}

Una posible mejora para obtener este matiz  ya que no se vieron aspectos como la relación entre los diferentes adjetivos o el contexto en el que se encuentran los mismos ni el peso de estos. Ya que es muy distinto el peso de un adjetivo como \emph{exelente} y un adjetivo como \emph{bueno}, siendo los dos adjetivos positivos.

%Anexos: dump de corridas, código y demás datos relevantes.
\section{Anexos}
%Anexos: dump de corridas, código y demás datos relevantes.

\subsection{Boludeces de latex}

Bla,blalb,a

\begin{flushleft}
Escribo sin tab
\end{flushleft}
Quiero enfatizar esta \emph{palabra}
\begin{itemize}
\item Item 1
\item Item 2
\subitem sub-item1
\subitem sub-item2
\item otro item
\end{itemize}

\subsection{Tabla Loca}

\begin{center} %Para que me quede centrada
\begin{tabularx}{0.97\linewidth}{XXXX} %ancho de linea, cantidad de columnas

Policía &	Lugar	&	Ladrón	&	País		\\
Dodero	&	Avión	&	Dedos	&	Alemania	\\
Elkin	&	Bar	&	Gato	&	Austria		\\
Frigerio&	Barco	&	Hurón	&	Espana		\\
Kesner	&	Cine	&	Rata	&	Francia		\\
Minari	&	Tren	&	Sombra	&	Inglaterra	\\

\end{tabularx}
\end{center}


\end{document}
