%%	SECCION documentclass																									 %%	
%%---------------------------------------------------------------------------%%
\documentclass[a4paper]{report}

%\usepackage{amsmath}
%\usepackage{amsfonts}
%\usepackage{amssymb}

%%---------------------------------------------------------------------------%%
%%	SECCION usepackage																											 %%	
%%---------------------------------------------------------------------------%%
\usepackage{amsmath, amsthm}
\usepackage{amsfonts}%
\usepackage{amssymb}%
\usepackage{caratula}
\usepackage{hyperref} %para que haya hiperv�nculos
\usepackage{graphicx} % Para el logo magico!
\usepackage{alltt} % que es este paquete???
\usepackage{tabularx}

\usepackage[spanish]{babel} 
\usepackage[T1]{fontenc}
\usepackage{textcomp}
\usepackage[utf8]{inputenc}

\pagenumbering{roman}
\hyphenation{biggest-satelit bi-gestsatelit biggestsa-telit biggestsate-lit}
\newenvironment{seccion}[1]{\section*{\underline{#1}}\linea \begin{flushleft}}{\end{flushleft}}

\oddsidemargin 0cm
\headsep -1.4cm
\textwidth=16.5cm
\textheight=23cm
%\makeindex

%%--------------------------------------------------------------------------
%% Nuevos Comandos
%% -------------------------------------------------------------------------

\newcommand{\comentario}[1]
{
\emph{\lq\lq{}#1\rq\rq{}}
}

%%---------------------------------------------------------------------------%%
%%	SECCION document	                                                       %%	
%%---------------------------------------------------------------------------%%
\begin{document}
	\setcounter{page}{0} %para que numere a la caratula como p�gina 0 y a las demas a partir de 1

%%---- Caratula -------------------------------------------------------------%%
\title{Trabajo Práctico IA2}
\author{Matías Pérez\\ Cecilia Sanchez}
\date{Agosto, 2010}

\materia{Inteligencia Artificial}
\titulo{Extracción de la Informacion}
\subtitulo{SEI-GO}
\fecha{31 Mayo, 2010}

\integrante{Cecilia Sanchez}{000/00}{chechus26@gmail.com}
\integrante{Matías Pérez}{002/05}{elmaildematiaz@gmail.com}
%\resumen{}

\maketitle

\tableofcontents

\newpage

\clearpage
%Introducción. Objetivo y alcance.
\section{Introduccion}

\subsection{Obejtivo}

El trabajo consiste en la realización de un sistema de extracción de información para la \emph{GuiaOleo}\footnote{http://www.guiaoleo.com.ar}, para poder conseguir información relevante a partir de los comentarios que se encuentran en la misma.

El criterio usado para definir si cierta información es relevante o no es si la misma consiste en apreciaciones sobre la comida, el servicio y otros aspectos de los restaurants. De esta forma se considera información irrelevante aquella que trate sobre la persona en sí u otros aspectos  que se suelen encontrar en los comentarios personales.

Vale aclarar que dicho criterio es arbitrario, ya que se hubiera podido elegir cualquier otro criterio.

\subsection{Alcance}

Lo que se intenta obtener con este sistema es una forma sencilla y rápida de conseguir las carácterísticas de los restaurants sin la necesidad de leer todos los comentarios. Se trata de obtener de los comentarios los fragmentos de los mismos que cumplen con el criterio ante dicho.

Por ejemplo, del comentario:

\begin{quotation}
Es uno de los pocos restaurantes que sirve cocina italiana de verdad y actualiza su menu; la atencion, si bien llevada mayormente por chicas jovenes es sobria, amable y eficiente. Voy muy seguido y he celebrado Navidad alli. Recomiendo tanto las pizzas como los platos de cocina. Dejen lugar para los postres...valen la pena acompañados por un ristretto lavazza autentico.
\end{quotation}

Se desea obtener \comentario{Recomiendo tanto las pizzas como los platos de cocina} y \comentario{la atencion, [..] es sobria, amable y eficiente}, ya que comentarios como \comentario{Voy muy seguido y he celebrado Navidad alli} no resultan relevantes ante el criterio elegido.

En este trabajo nos conformaremos con que sólo reconozca la primera oración, ya que la segunda contiene información no relevante en el medio de la oración. Dejando el reconocimento de este tipo de oraciones para etapas posteriores y futuros refinamientos.
%Descripción y desarrollo de los experimentos.
\section{Descripcion}
%Descripción y desarrollo de los experimentos.

El dasarrollo del sistema se puede dividir en cuatro etapas: \emph{armado del diccionario}, \emph{definicion de reglas}, \emph{aplicacion de reglas} y \emph{recoleccion de resultados}. Las mismas serás descriptas en más profundidas a continuación.

\subsection{Armado del diccionario}

Antes de empezara a analizar los comentarios, fué nescesario definir las palabras que definen nuestro mundo y caracterizarlas, ya que por ejemplo, no cumplen la misma función gramátical \emph{pizza} o \emph{exelente}.

Las entidades gramáticales que se terminaron usando son:
\begin{itemize}
\item Entidad: Define a los sustantivos del dominio relevantes, pueden ser comidas o mesas, ambiente, etc.
\item Adjetivo: Son los adjetivos calificativos, estos a su vez se dividieron en:
\subitem Positivo
\subitem Neutro
\subitem Negativo
\item Verb: verbos
\item Art: articulos
\item Empty: Son todas aquellas palabras que no están definidas como ninguna otra entidad gramátical.
\end{itemize}

En este punto, hubiera sido ideal poder definir todas las palabras del idioma castellano, pero esto no es posible, por lo que se hizo fué definir las palabras más relevantes en nuestro dominio (los comentarios de la guia oleo).

Con este fin fué nesesario primero determinar cuáles son estas palabras, lo que se optó por hacer fué simplemente ordenar las palabras por cantidad de apariciones en los comentarios. Con este procedimiento se logró conseguir una primera aproximación de las palabras más relevantes.

Pero dado que el tiempo con que se contaba no era suficiente para caracterizar todas las palabras, se optó por caracterizar las palabras por su raíz, esto disminuyó notablemente la cantidad de palabras, ya que para caracterizar palabras como \emph{malo, mala, mal, malos, malas, males} sólo fué nescesario caracterizar \emph{mal}.

Este método si bien ayuda a disminuir la cantidad de elementos del dominio, también trae sus consecuencias, ya que palabras como \emph{precios} y \emph{preciosamente} tienen la misma raíz \emph{preci}, siendo la primera una \emph{Entidad} y la otra un \emph{Adjetivo}. Pero esto fué algo con lo que se aceptó seguir adelante.

Una vez obtenido un diccionario con todas las raices ordenadas por orden de aparición se comenzaron a caracterizar las más importantes manualmente. Es necesario aclarar que la caracterización de las palabras fué considerando el contexto de la \emph{GuíaOleo}, ya que palabras como \emph{tierno} se tomó como un adjetivo positivo y palabras como \emph{abuelo} no se consideraron como entidades a pesar de ser sustantivos.

Luego de completar las subsiguientes etapas, una vez que se obteubo un sistema andando con este diccionario, se decidió enriquecer el diccionario de un modo más automático. Este proceso será explicado más adelante.


\subsection{Definición de reglas}

Una vez obtenido el diccionario de palabras a usar, el siguiente paso fué representar el criterio por el cuál se definiría si una frase es relevante o no. Esto se realizó mediante el uso de expresiones regulares.

De esta manera se definieron expresiones regulares de modo tal que sólo las frases que cumplan con estas expresiones cumplan con el criterio propuesto. Esto fué hecho manualmente y las expresiones regulares se fueron creando a base de sentido común y pruebas en el conjunto de datos de entrenamiento.

Las reglas que se obtubieron finalmente fueron las siguientes:

\begin{verbatim}
(ART)* ENTITY (EMPTY)+ (ADJETIVE)+ ( y |ADJETIVE)*
(ART)* ENTITY (VERB)+ (ADJETIVE)+ (EMPTY|ADJETIVE)*
(ART)* ENTITY (VERB)+ (EMPTY)+ (ADJETIVE)+ (EMPTY|ADJETIVE|ART)* (ENTITY)*
(ART)+ (ADJETIVE)+ ENTITY (EMPTY)+ (ADJETIVE)+
(ART)+ ENTITY (ADJETIVE)+
(no)* (VERB)* (ART)+ (ADJETIVE)+ ENTITY
(ART)* ADJETIVE ART ENTITY
(ART)* VERB ADJETIVE ART ENTITY
VERB VERB ADJETIVE
Muy ADJETIVE (ART)* ENTITY
uno de los ADJETIVE ENTITY para comer
no lo recomiendo
no es para recomendar
\end{verbatim}

Se puede ver que la mayoría de las reglas depende sólo de las entidades definidas, pero hay casos en los que la expresion de la caracterización no es suficiente y se pierden casos de gran interés, como ser \comentario{no lo recomiendo}. Para no perder esto simplemente se agregó una regla que busca esta frase en particular.

En este punto es justo hacer notar que los modificadores tales como \emph{muy}, \emph{no}, \emph{más}, etc. no fueron caracterizados, haciendo imposible la detección de estas palabras en frases más complejas  Al igual que no se detectan frases que contengan aposiciones u otras construcciones gramáticas con relativo grado de complejidad. Estos casos quedan para futuras iteraciones de este sistema.

\subsection{Aplicación de reglas}

Una vez que se cuenta con el diccionario de las palabras ya caracterizado y con las reglas definidas, es necesario aplicar estas reglas a los comentarios. Con este fín se desarrolló un sistema informático que simplemente busca la aparicion de las expresiones regulares en un comentario.

El mismo toma los diferentes archivos con cada tipo de palabras, un archivo con las reglas definidas y por último los comentarios a evaluar.

Este devuelve un archivo de texto con todos los comentarios seguidos de las frases obtenidas y un detalle de cuál fué la regla que se usó para detectar la frase.

\subsection{Recolección de resultados}
% Positivos, negativos, bla...


\subsection{Consideraciones finales}
% Be free ;)

%Metodologías y métricas utilizadas.
\section{Metodología}

\subsection{Sub-seccion1}
Bla,blalb,a
\subsection{Sub-seccion2}
BlaBla
\begin{flushleft}
Escribo sin tab
\end{flushleft}
Quiero enfatizar esta \emph{palabra}
\begin{itemize}
\item Item 1
\item Item 2
\subitem sub-item1
\subitem sub-item2
\item otro item
\end{itemize}










%Presentación y análisis de resultados.
\section{Resultados}
%Presentación y análisis de resultados.
%Conclusiones.
\section{Conclusiones}

Los resultados obtenidos por el sistema implmentados fueron más que satisfactorios. Se puedo apreciar cómo con simples reglas se puede identificar, a apartir de texto libre en lenguaje natural, contenidos relevantes de contenidos fútiles bajo cierto criterio.

Cuando se comnzó con el trabajo de investigación no se tenían grandes expectativas para con el mismo. Pero a medida que el mismo fué creciendo y perfeccionandose se pudo ver cómo se puede crear un sistema funcional que cumpla con el propósito planteado. A medida que se fué iterando tanto sobre las reglas que definen el criterio de relevancia de una frase, como sobre el armado del diccionario de las palabras del dominio se pudo ver como estas refinaciones perfeccionaron el mismo. El progreso del sistema se vió plasmado claramente en los resultados que se fueron viendo.

En las primeras iteraciones a partir de un comentario como:

\begin{quotation}
\emph{
\lq\lq{}Para redondear, todo es de mal gusto, salvo la camarera que nos atendio bien, el resto desastre, precios altisimos, la comida no es mas de lo que se puede comer en un bodegon italiano, recomiendo que no vayan, van a salir de mal humor.\rq\rq{}
}

\end{quotation}

Se obtenían resultados como:

\begin{quotation}
\begin{itemize}
\item R0  bodegon italiano , recomiendo
\item R3  el resto desastre
\end{itemize}
\end{quotation}

Luego, al ir iterando sobre el sistema y a medida que se detectaron las falencias del mismo fué posible que a partir del mismo comentario el sistema detectara correctamente las oraciones relevantes. En el caso planteado anteriormente luego de refinar las reglas y el diccionario fué posible identificar como irrelevante \comentario{bodegon italiano , recomiendo}, con lo que el resultado obtenido finalmente fué sólo el siguiente:

\begin{quotation}
\begin{itemize}
\item R4  el resto desastre
\end{itemize}
\end{quotation}

Veamos otro ejemplo; en una de las primeras iteraciones, ante el siguiete comentario:

\begin{quotation}
\emph{
\lq\lq{} {\bf La pizza nos encanta}, aunque esta vez vino medio fria y bastante quemada. El servicio de mesa es caro (para dos bruschettas de tomate). {\bf La bebida, carisima} y los postres, tambien {\bf son muy caros} dada la calidad. En fin, {\bf la pizza es excelente}, lo demas, deja bastante que desear. {\bf EL servicio es pesimo}, y bastante lento: le pedimos a la moza que nos limpiara, por favor, {\bf la mesa y lo hizo de muy mala gana}. El mantenimiento de los baños es poco.\rq\rq{}
}

\end{quotation}

Se obtenían resultados como:

\begin{quotation}
\begin{itemize}
\item R0  La bebida , carisima 
\item R0 son muy caros
\item R0 la mesa y lo hizo de muy mala
\item R1 La pizza nos encanta
\item R1 mesa es caro
\item R1 la pizza es excelente
\item R1 El servicio es pesimo
\end{itemize}
\end{quotation}

Pero luego se consiguió oibtener el siguiente resultado:

\begin{quotation}
\emph{
\lq\lq{} {\bf La pizza nos encanta}, aunque esta vez vino medio fria y bastante quemada. El servicio de {\bf mesa es caro} (para dos bruschettas de tomate). La bebida, carisima y los postres, tambien son muy caros dada la calidad. En fin, {\bf la pizza es excelente}, lo demas, deja bastante que desear. {\bf EL servicio es pesimo}, y bastante lento: le pedimos a la moza que nos limpiara, por favor, la mesa y lo hizo de muy mala gana. El mantenimiento de los baños es poco.\rq\rq{}
}

\end{quotation}

Obteniendo resultados más precisos, que es el objetivo que se planteó. Ya que como se explicó en un comienzo se prioriza no identificar información irrelevante como relevante, a riesgo de obviar información que resulte relevante.

Aunque se puede notar que el sistema no es perfecto, sí se puede decir que considerando la totalidad de los casos, el mismo identifica correctamente las partes relevantes en las mayorías de estos.

Fuera de las iteraciones que se realizaron hasta el momento del sistema quedan casos que tiene que ver con la gramática, el contexto o frases verbales. De igual manera resulta interesante ver cómo sólo teniendo en cuenta las oraciones en sí y su configuración se pueden lograr resultados ampliamente satisfactorios, como ser:

\begin{quotation}
\emph{
\lq\lq{}
Filo continua sirviendo platos de {\bf la cocina italiana de muy buena calidad}: {\bf la pizza estaba deliciosa}; el rissoto hecho en el momento, con funghi de primera y arroz arborio, estaba  impecable; el semifredo final fue un poema.  {\bf La atencion muy buena}, especialmente en un lugar con tanto movimiento como es Filo, {\bf La decoracion descontracturada} y transgresora muy acorde con el sitio.  En resumen un restaurante de {\bf comida italiana de muy buen nivel} en el centro de BA.  Para mi gusto, {\bf un lugar muy recomendable}, que me da la seguridad de encontrar en el lo que busco cuando quiero comida italiana. 
\rq\rq{}
}
Resultado:
\begin{itemize}
\item R0 la cocina italiana de muy buena calidad
\item R0 la pizza estaba deliciosa
\item R0 la atencion muy buena
\item R0 comida italiana de muy buen nivel
\item R0 un lugar muy recomendable
\item R4  la decoracion descontracturada
\end{itemize}
\end{quotation}



%Trabajo Futuro y Referencias.
\section{Trabajo Futuro}

%Anexos: dump de corridas, código y demás datos relevantes.
\section{Anexos}
%Anexos: dump de corridas, código y demás datos relevantes.

\subsection{Boludeces de latex}

\emph{Setear la codificación del editor UTF-8, es la default... pero si no se ven bien lo acentos o algo es por eso.}

Bla,blalb,a
\begin{quotation}
Acá puedo poner un comentario. lo identa locamente :P
\end{quotation}

\begin{flushleft}
Escribo sin tab
\end{flushleft}
Quiero enfatizar esta \emph{palabra}
\begin{itemize}
\item Item 1
\item Item 2
\subitem sub-item1
\subitem sub-item2
\item otro item
\end{itemize}

\subsection{Tabla Loca}

\begin{center} %Para que me quede centrada
\begin{tabularx}{0.97\linewidth}{XXXX} %ancho de linea, cantidad de columnas

Policía &	Lugar	&	Ladrón	&	País		\\
Dodero	&	Avión	&	Dedos	&	Alemania	\\
Elkin	&	Bar	&	Gato	&	Austria		\\
Frigerio&	Barco	&	Hurón	&	Espana		\\
Kesner	&	Cine	&	Rata	&	Francia		\\
Minari	&	Tren	&	Sombra	&	Inglaterra	\\

\end{tabularx}
\end{center}


\end{document}
