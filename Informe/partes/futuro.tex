\section{Trabajo Futuro}

A continuación se detallaran algunas ideas que surgieron en el
desarrollo del trabajo pero que no se llevaron finalmente a cabo, dejandolas para
próximas versiones.


\subsection{Expresiones Regulares:}

Para la evaluación de las expresiones regulares se realiza el siguiente proceso:
Se evalúa la oración con todas las reglas, por cada una de estas se
recorre las palabras, si estas coinciden con la estructura que se
espera se la guarda como resultado y se sigue evaluando la regla con el
resto de la frase.

Esto se lleva a cabo con un modulo que implementa
la evaluación de expresiones regulares sobre texto, una idea
interesante es etiquetar toda la oración con las palabras claves de
los distintos grupos y aplicarle una expresión regular. Esto viene
implementado en el lenguaje, el problema es que una vez que se
obtiene qué parte de la oración de palabras claves cumple con la
expresión regular, resulta dificultoso encontrar la parte de la oración
original a la que hace referencia. Esta mejora puede hacer que se
tengan reglas mas complejas.


\subsection{Adjetivos:}

Para clasificar a los comentarios se cuentan la cantidad de adjetivos
positivos y negativos encontrados luego de evaluar las reglas; cada
uno de estos tiene peso de un voto, pero esta claro que no es lo
mismo decir excelente que horrible y lindo. Por esta razón se puede
asignarle un peso diferente a los adjetivos y tener una aproximación
mas correcta del matíz del comentario, para la asignación de pesos
existe un indicador que se llama \emph{PMI} que indica la \lq\lq{}sinominia\rq\rq{} de las
palabras definida sobre la probabilidad. Esto no fue implementado dado
que en el dominio del problema la votación funciona con un alto
porcentaje de aciertos.

\subsection{Interfaz:}

Otra idea que no se llegó a implementar se centra en la interfaz del sistema.
Dado que la guia oleo consiste una página electrónica hubiera sido interezante acoplar el sistema desarrollado con un parser html que a partir de la página original genere una página igual, pero con las oraciones relevantes remarcadas de algún modo. Así también se podría proveer de un resumen del restaurant en particilar.

Haciendo que la utilización del sistema sea mucho más intuitiva y integrandolo a la página original.
