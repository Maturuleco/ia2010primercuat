\section{Introduccion}

\subsection{Obejtivo}

El trabajo consiste en la realización de un sistema de extracción de información para la \emph{GuiaOleo}\footnote{http://www.guiaoleo.com.ar}, para poder conseguir información relevante a partir de los comentarios que se encuentran en la misma.

El criterio usado para definir si cierta información es relevante o no es si la misma consiste en apreciaciones sobre la comida, el servicio y otros aspectos de los restaurants. De esta forma se considera información irrelevante aquella que trate sobre la persona en sí u otros aspectos  que se suelen encontrar en los comentarios personales.
Vale aclarar que dicho criterio es arbitrario, ya que se hubiera podido elegir cualquier otro criterio.

\subsection{Alcance}

Lo que se intenta obtener con este sistema es una forma sencilla y rápida de conseguir las carácterísticas de los restaurants sin la necesidad de leer todos los comentarios. Se trata de obtener de los comentarios los fragmentos de los mismos que cumplen con el criterio ante dicho.

Por ejemplo, del comentario:

\begin{quotation}
Es uno de los pocos restaurantes que sirve cocina italiana de verdad y actualiza su menu; la atencion, si bien llevada mayormente por chicas jovenes es sobria, amable y eficiente. Voy muy seguido y he celebrado Navidad alli. Recomiendo tanto las pizzas como los platos de cocina. Dejen lugar para los postres...valen la pena acompañados por un ristretto lavazza autentico.
\end{quotation}

Se desea obtener \comentario{Recomiendo tanto las pizzas como los platos de cocina} y \comentario{la atencion, [..] es sobria, amable y eficiente}, ya que comentarios como \comentario{Voy muy seguido y he celebrado Navidad alli} no resultan relevantes ante el criterio elegido.

En este trabajo nos conformaremos con que sólo reconozca la primera oración, ya que la segunda contiene información no relevante en el medio de la oración. Dejando el reconocimento de este tipo de oraciones para etapas posteriores y futuros refinamientos.